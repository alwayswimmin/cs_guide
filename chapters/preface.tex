\chapter{Preface}

You might have heard of \href{https://www.dropbox.com/s/z8qdndxmrmxqsam/Napkin.pdf?oref=e&n=97419869}{Evan Chen's Napkin}, a resource for olympiad math people that serves as a jumping point into higher mathematics.\footnote{In fact, I'm using Evan's template right now. Thanks Evan!} The Wikipedia articles on higher mathematics are just so dense in vocabulary and deter many smart young students from learning them before they are formally taught in a course in college. Evan's Napkin aims to provide that background necessary to leap right in.

I feel the same way about computer science. For most, the ease of the AP Computer Science test means that the AP coursework is often inadequate in teaching the simplest data structures, algorithms, and big ideas necessary to approach even silver USACO problems. On the other hand, even the best reference books, like Sedgewick, are too dense and unapproachable for someone who just wants to sit down and learn something interesting.\footnote{Sedgewick, notably, is getting better. \href{http://algs4.cs.princeton.edu/home/}{Check out his online companion} to \textit{Algorithms, 4th Edition}.} The road, for many, stalls here until college. Everyone should be able to learn the simplest data structures in Java or C++ standard libraries, and someone with problem-solving experience can easily jump right into understanding algorithms and more advanced data structures.

A few important notes, before we begin.

\begin{itemize}

\item

I'm assuming some fluency in C-style syntax. If this is your first time seeing code, please look somewhere else for now.

\item

It is essential that you understand the motivations and the complexities behind everything we cover. I feel that this is not stressed at all in AP Computer Science and lost under the heavy details of rigorous published works. I'm avoiding what I call the heavy details because they don't focus on the math behind the computer science and lose the bigger picture. My goal is for every mathematician or programmer, after working through this, to be able to code short scripts to solve problems. Once you understand how things work, you can then move on to those details which are necessary for building larger projects. The heavy details become meaningless as languages develop or become phased out. The math and ideas behind the data structures and algorithms will last a lifetime.

\item

It is recommended actually code up each data structure with its most important functions or algorithm as you learn them. I truly believe the only way to build a solid foundation is to code. Do not become reliant on using the standard library (\texttt{java.util}, for instance) without understanding how the tool you are using works.

\end{itemize}

