\chapter{Big Ideas}

In this chapter, we'll discuss some general problem solving ideas: brute force, depth-first search, and the greedy algorithm. We can think of these as the building blocks to more complex methods---each provides a very general approach to simplifying problems. In programming contests, they also appear frequently by themselves as the core ideas to solutions. Since the concepts we cover are independent of language, we will no longer present algorithms in concrete Java or C++ code, but rather in more abstract pseudocode.

\section{Brute Force}

Sometimes, the best way to approach a problem is to try everything. This idea of exhaustively searching all possibilities is called \emph{brute force}. For example, if we want to unlock a friend's iPhone, we could try all of the $10^4$ possible passcodes. As the name and this example suggest, brute force is often crude and inefficient. Usually we want to make some clever observations to make the problem more tractable. However, if the input size is small (check the number of operations against $10^8$) or if we want to squeeze a few points out of a problem by solving only the small cases, brute force could be the way to go. And if you're stuck on a problem, thinking about a brute force is not a bad way to start. Simpler, slower algorithms can often inspire faster ones. Through the following problems, we'll show you how to brutally apply the idea of brute force.

\subsection{Square Root}

\begin{typewriter}
  Given an integer $n$, $1 \le n \le 10^{12}$, find the greatest integer less than or equal to $\sqrt{n}$ without using any library functions. (This means you can't call functions like Math.sqrt or Math.log.)
\end{typewriter}

At first, it's not obvious how we can compute square roots. However, we can always go simple. Set $i = 1$, and while $(i+1)^2\le n$, increment $i$. That is, we increment $i$ until increasing it further will cause $i$ to exceed $\sqrt n$. Since our answer $i$ is at most $\sqrt n \le 10^6$, our program runs in time. This is about the silliest approach we can use to calculate square roots, but hey, it works!

When implementing this algorithm, be careful about the size of $n$. The 32-bit \texttt{int} type in Java and C++ only holds values up to $2^{31} - 1 = 2,147,483,647$, which is exceeded by the maximum possible value of $n$. Thus we need to use a 64-bit integer type for our calculations: \texttt{long} in Java and \texttt{long long} in C++.

\subsection{Combination Lock}

\begin{typewriter}
  Farmer John purchases a combination lock to stop his cows from escaping their pasture and causing mischief! His lock has three circular dials, each with tick marks numbered $1$ through $N$ ($1\le N\le 100$), with $1$ and $N$ adjacent. There are two combinations that open the lock: one combination set by Farmer John and one ``master'' combination set by the locksmith. The lock has a small tolerance for error, however, so it will open if the numbers on each of the dials are at most two positions away from that of a valid combination. Given Farmer John's combination and the master combination, determine the number of distinct settings for the dials that will open the lock. 

  (For example, if Farmer John's combination is $(1,2,3)$ and the master combination is $(4,5,6)$, the lock will open if its dials are set to $(1,N,5)$ (since this is close to Farmer John's combination) or to $(2,4,8)$ (since this is close to the master combination). Note that $(1,5,6)$ would not open the lock, since it is not close enough to any single combination. Furthermore, order matters, so $(1,2,3)$ is distinct from $(3,2,1)$.) [Adapted from \href{http://usaco.org/index.php?page=viewproblem2&cpid=340}{USACO 2013, Combination Lock}.]
\end{typewriter}

Again, the simplest idea works. We can iterate over all possible settings of the lock, and for each setting, check if it matches either Farmer John's combination or the master combination. To do this, we can use three nested \texttt{for} loops. The first loop goes through the values for the first dial, the second loop through the values for the second dial, and the third loop through the values for the third dial. Since there are three dials, the lock has at most $N^3 \le 10^6$ possible settings. We can check if each dial matches in $O(1)$ time, hence our algorithm runs in less than a second.

In terms of implementation, \texttt{Combination Lock} is a great example of how a problem can decompose into two easier components that we can think about separately. The first component is to use nested loops to iterate through the possible settings, which we've described above. (Nested \texttt{for} loops like this show up often!) The second component is to check if a given setting is close to either of the given combinations. If we implement a function \texttt{is\_valid(a, b, c)} to do this, then the code becomes quite clean.

\subsection{Ski Course Design}

\begin{typewriter}
Farmer John has $N$ hills on his farm ($1\le N\le 1000$), each with an integer elevation in the range $0$ to $100$. In the winter, since there is abundant snow on these hills, he routinely operates a ski training camp. In order to evade taxes, Farmer John wants to add or subtract height from each of his hills so that the difference between the heights of his shortest and tallest hills is at most $17$ before this year's camp.

Suppose it costs $x^2$ dollars for Farmer John to change the height of a hill by $x$ units. Given the current heights of his hills, what is the minimum amount that Farmer John will need to pay? (Farmer John is only willing to change the height of each hill by an integer amount.) [Adapted from \href{http://usaco.org/index.php?page=viewproblem2&cpid=376}{USACO 2014, Ski Course Design}.]
\end{typewriter}

For \texttt{Ski Course Design}, we need to be a bit clever about how to implement our brute force. There are infinitely many ways we could change the heights of each hill, so it seems intractable to iterate over the possible heights for each hill separately. Instead, we look at the final range of the ski slope heights, which has length at most $17$. This final range has to fall within the interval $[0,100]$, hence there are less than $100$ possibilities. (The possible ranges are $[0, 17]$, $[1, 18]$, $\cdots$, $[83, 100]$.) Once we fix a range, we can calculate in $O(N)$ the minimum cost to make the height of each hill fall within that range. Thus if we let $M$ be the number of possible ranges ($M < 100$), we have an $O(MN)$ algorithm.

This problem shows that even when we brute force, we still have to think. Some approaches are better than others. In particular, we don't want to deal with cases that are irrelevant---for example, when the heights are not within a range of width $17$ or when Farmer John has not used the cheapest set of changes. We also don't want our possibilities to explode out of control, which would have happened had we adjusted the height of each hill separately with nested \texttt{for} loops or recursion. By iterating over a more restricted set of possibilities, we have created a brute force solution that runs significantly faster.

\subsection{Contest Practice}

\href{http://codeforces.com/group/5tN48zOVvQ/contest/204642}{Here} is a collection of problems solvable through brute force. Try working through them on your own and applying the ideas you've seen so far. (May the brute force be with you.)

\section{Depth-First Search (DFS)}

Depth-first search is a recursive search technique that provides us with another way to brute force. Instead of using nested loops to iterate through all possibilities, we can generate the possibilities using recursion, checking all possible choices in each level. Here's an example: All of your friends, upon realizing that it only takes four nested \texttt{for} loops to iterate through all possible 4-digit iPhone passcodes, have decided to make their passcodes $n$ digits long. Since you don't know $n$, nested \texttt{for} loops will no longer do the trick. Instead, we can use a DFS to recursively generate all $n$-digit passcodes.

Depth-first search works as follows: We check all passcodes starting with ``0'', then all passcodes starting with ``1'', then all passcodes starting with ``2'', and so on. To check all passcodes starting with ``0'', we check all passcodes starting with ``00'', then all passcodes starting with ``01'', then all passcodes starting with ``02'', and so on. To check all passcodes starting with ``00'', we have to check all passcodes starting with ``000'', then all passcodes starting with ``001'' and so on... (Think about why DFS is \emph{depth-first}.)

In this way, we recursively generate all possible passcodes by extending the prefix character by character. We keep recursing until we have to check a passcode starting with a string of length $n$, in which case that string is the passcode itself. Thus the first passcode we check is ``00$\cdots$0'' and the last passcode we check is ``99$\cdots$9''. We implement this algorithm by writing a function that generates all passcodes given a prefix. Below is some pseudocode describing the algorithm. Make sure you understand how the function calls itself!

\noindent \begin{minipage}{\textwidth}
  \begin{algorithmic}[1]
    \Function{generatePasscodes}{$depth$, $prefix$}
    \If{$depth = n$}
      \Comment {If we've reached maximum depth, then print and return.}
      \State \Call{print}{$prefix$}
      \State \Return
    \EndIf
    \For{$c$ from `0' to `9'}
      \Comment{Iterates over all possible next digits.}
      \State \Call{generatePasscodes}{$depth + 1$, $prefix + c$}
      \Comment{Recurses with a longer prefix.}
    \EndFor
    \EndFunction
  \end{algorithmic}
\end{minipage}

\subsection{Permutations}

\begin{typewriter}
  Given $n$ ($n \le 8$), print all permutations of the sequence $\{1, 2, \cdots, n\}$ in lexicographic (alphabetical) order. (For $n=3$, this would be $(1, 2, 3)$, $(1, 3, 2)$, $(2, 1, 3)$, $(2, 3, 1)$, $(3, 1, 2)$, and $(3, 2, 1)$.)
\end{typewriter}

Like the passcode problem, we use DFS instead of nested \texttt{for} loops, since we don't know $n$. However, we have to be careful with implementation---we can use each number only once. Along with our current prefix, we have to keep track of the set of numbers that we've already used. This is best done with a Boolean ``used'' array outside of the recursive function. Here's the pseudocode:

\noindent \begin{minipage}{\textwidth}
  \begin{algorithmic}[1]
    \State $used \gets \{false, false, \cdots, false\}$
    \Comment{Initialize $used$ as an array of $false$ values.}
    \Function{generatePermutations}{$depth$, $prefix$}
    \If{$depth = n$}
      \State \Call{print}{$prefix$}
      \State \Return
    \EndIf
    \For{$i=1$ to $n$}
      \If{not $used[i]$}
        \State $used[i] \gets true$
        \State \Call{generatePermutations}{$depth + 1$, $prefix + i$}
        \State $used[i] \gets false$
        \Comment We have to reset the $used[i]$ variable once we're done.
      \EndIf
    \EndFor
    \EndFunction
  \end{algorithmic}
\end{minipage}

To understand the order in which we visit the permutations, we can visualize this algorithm as traversing a tree-like structure. An animation of this algorithm for $n = 5$ is \href{http://dabbler0.github.io/ecc-animations/dfs.html}{here}.

\subsection{Basketball}

\begin{typewriter}
Two teams are competing in a game of basketball: the Exonians and the Smurfs. There are $n$ players on the Exonian team and $m$ players on the Smurf team, with $n + m \le 17$. Each player has an integer skill level $s$ between $1$ and $10^8$. Define the strength of a set of players as the sum of their individual skill levels. In order to ensure a fair game, the Exonians and Smurfs plan on choosing two equally strong starting lineups. In how many ways can the two teams choose their lineups? (Two lineups are considered different if there exists a player who starts in one game, but not in the other.)
\end{typewriter}

We use a DFS to recursively generate all possible starting lineups. Each starting lineup can be represented by a sequence of $n + m$ 0's and 1's, where a player starts if and only if he/she is assigned a 1. We do this the same way we generate all passcodes of length $n + m$. Once we have a starting lineup, it is straightforward to check for fairness. (Is it also possible to keep track of the strength of each team as we DFS? Hint: Keep an extra variable similar to ``used'' in \texttt{Permutations}.)

\subsection{Problem Break}

Before moving on, try to implement the DFS problems described above. You can test your code on the problem set \href{http://codeforces.com/group/5tN48zOVvQ/contest/205012}{here}. Try to do some complexity analysis too. How does the runtime of these algorithms grow with respect to $n$?

\subsection{Generalizing DFS}

Thus far, all of the DFS solutions that we've seen have involved sequences. However, we can also use DFS in a much more general setting. In the same spirit as brute force, if we want to enumerate or construct something and we have a number of options at each step, we can recurse on each of the possible options and thereby check all possibilities. The examples below show the flexibility of depth-first search---a huge class of problems can be solved in a brute force manner like this.

\subsection{Dungeon}

\begin{typewriter}
  Bessie is trying to escape from the dungeon of the meat packing plant! The dungeon is represented by an $n$-by-$n$ grid ($2 \le n \le 6$) where each of the grid cells is trapped and can only be stepped on once. Some cells of the grid also contain obstacles that block her way. Bessie is currently located in the upper-left corner and wants to make her way to the exit in the lower-right corner. How many paths can Bessie take to escape, assuming that she avoids all obstacles and steps on no cell twice?
\end{typewriter}

We write a function \texttt{DFS(x, y)} thats runs a DFS from cell $(x, y)$ and counts the number of paths to the lower-right corner given obstacles and previously visited squares. Upon arriving at $(x, y)$, we mark that cell as visited. If $(x, y)$ is the destination, we increment our answer by one. Otherwise, we try recursing in each direction---up, down, left, and right. Before recursing, we check that we don't go out of bounds and that we don't step on an obstacle or previously visited square. Once we're done counting paths from $(x, y)$, we have to remember to mark this cell as unvisited again.

In terms of implementation, it is easiest to store the obstacles and visited cells in Boolean arrays outside of the recursive function. We can also define a function \texttt{ok(x, y)} to check if a cell $(x, y)$ is safe to step on---if $x$ and $y$ are in bounds, $(x, y)$ is not an obstacle, and $(x, y)$ is unvisited. Finally, to avoid doing four cases, one for each direction, we can define two arrays \texttt{dx} and \texttt{dy} which contain \texttt{[1, 0, -1, 0]} and \texttt{[0, 1, 0, -1]}, respectively. Then \texttt{dx[i]} and \texttt{dy[i]} for $0 \le i < 4$ represent the changes in $x$- and $y$-coordinates for each of Bessie's possible moves.
\subsection{$n$ Queens Puzzle}

\begin{typewriter}
  Given $n$ ($4 \le n \le 8$), find an arrangement of $n$ queens on an $n$-by-$n$ chessboard so that no two queens attack each other. 
\end{typewriter}

First, observe that each row must contain exactly one queen. Thus we can assume that the $i$-th queen is placed in the $i$-th row. Like before, we try putting each queen on each possible square in its row and recurse, keeping track of used columns and diagonals. When we add the $i$-th queen, we mark its column and diagonals (both of them) as attacked. As usual, once we're done checking all possibilities for the $i$-th queen, we unmark its column and diagonals. We terminate when we find an arrangement using all $n$ queens.

\section{Greedy Algorithms}

A greedy algorithm is an algorithm that makes the (locally) most desirable choice at each step. Instead of checking all possibilities as we do in a depth-first search, we choose the option that ``seems'' best at the time. This usually results in a fast and easy to implement algorithm. While this may not be a good way to live life---doing homework would always be relegated in favor of solving programming contest problems---there exist many computer science problems where being greedy produces optimal or almost optimal results. Keep in mind, however, that greedy algorithms usually don't work. Before coding any greedy solution, you should be able to convince yourself with a proof of correctness. Let's go through a few examples to get a better sense for what greedy algorithms are like.

\subsection{Bessie the Polyglot}

\begin{typewriter}
  Bessie would like to learn how to code! There are $n$ ($1\le n\le 100$) programming languages suitable for cows and the $i$-th of them takes $a_i$ ($1\le a_i\le 100$) days to learn. Help Bessie determine the maximum number of languages she could know after $k$ ($1\le k\le 10^4$) days. [Adapted from \href{http://codeforces.com/problemset/problem/507/A}{Codeforces 507A}.]
\end{typewriter}

It's pretty easy to see that we want to learn programming languages in order of increasing $a_i$, starting from the smallest. Thus Bessie can obtain an optimal solution with a greedy approach. She first learns the language that requires the fewest days to learn, then the language that takes the next fewest days, and so on, until she can't learn another language without exceeding $k$ days. To implement this, we sort $a_i$ and find the longest prefix whose sum is at most $k$. Due to sorting, the complexity is $O(n\log n)$.

With greedy algorithms, we usually want to have an ordering or heuristic by which we decide what is the best choice at a given instance. In this case, our ordering was the most straightforward possible, just choosing languages in order of increasing learning time. However, figuring out how we want to greedily make decisions is not always obvious. Oftentimes, coming up with the correct heuristic is what makes greedy algorithms hard to find.

\subsection{More Cowbell}

\begin{typewriter}
  Kevin Sun wants to move his precious collection of $n$ ($1\le n\le 10^5$) cowbells from Naperthrill to Exeter, where there is actually grass instead of corn. Before moving, he must pack his cowbells into $k$ boxes of a fixed size. In order to keep his collection safe during transportation, he will not place more than two cowbells into a single box. Since Kevin wishes to minimize expenses, he is curious about the smallest size box he can use to pack his entire collection.

  Kevin is a meticulous cowbell collector and knows that the size of his $i$-th ($1\le i\le n$) cowbell is an integer $s_i$. In fact, he keeps his cowbells sorted by size, so $s_{i - 1}\le s_i$ for any $i > 1$. Also an expert packer, Kevin can fit one or two cowbells into a box of size $s$ if and only if the sum of their sizes does not exceed $s$. Given this information, help Kevin determine the smallest $s$ for which it is possible to put all of his cowbells into $k$ ($n\le 2k\le 10^5$) boxes of size $s$. [Adapted from \href{http://codeforces.com/problemset/problem/604/B}{Codeforces 604B}.]
\end{typewriter}

Intuitively, we want to use as many boxes as we can and put the largest cowbells by themselves. Then, we want to pair the leftover cowbells so that the largest sum of a pair is minimized. This leads to the following greedy algorithm:

First, if $k \ge n$, then each cowbell can go into its own box, so our answer is $\max(s_1, s_2, \cdots, s_n)$. Otherwise, we can have at most $2k-n$ boxes that contain one cowbell. Thus we put the $2k-n$ largest cowbells into their own boxes. For the remaining $n-(2k-n) = 2(n-k)$ cowbells, we pair the $i$th largest cowbell with the $(2(n-k) - i + 1)$-th largest. In other words, we match the smallest remaining cowbell with the largest, the second smallest with the second largest, and so on. Given these pairings, we can loop through them to find the largest box we'll need. The complexity of this algorithm is $O(n)$ since we need to iterate through all of the cowbells.

To prove that this greedy algorithm works, we first prove the case $n = 2k$, where each box must contain exactly two cowbells. Consider any optimal pairing of cowbells. If $s_{2k}$ is not paired with $s_1$, then we can perform a swap so that they are paired without increasing the size of the largest box: Suppose we initially have the pairs $(s_1, s_i)$ and $(s_{2k}, s_j)$. If we rearrange them so that we have the pairs $(s_1, s_{2k})$ and $(s_i, s_j)$, then $s_1 + s_{2k}, s_i + s_j \le s_{2k} + s_j$. After we've paired the largest cowbell with the smallest, we can apply the same logic to the second largest, third largest, and so on until we're done. Therefore, our construction is optimal if $n = 2k$. For $n < 2k$, we can imagine that we have $2k - n$ cowbells of size $0$ and use the same argument.

This method of proving correctness for a greedy algorithm is rather common. We want to show that our greedily constructed solution is as good as an arbitrarily chosen optimal solution, so we compare where the optimal solution and our greedy solution differ. Once we find a difference, we try to transform one to the other without changing the value we're trying to optimize. In this case, since transforming the optimal solution to our greedily constructed solution doesn't make it worse, our solution must be optimal as well.

\subsection{Farmer John and Boxes}

\begin{typewriter}
  Farmer John has $n$ ($1 \le n \le 100$) boxes in his barn. His boxes all have the same size and weight but have varying strengths---the $i$-th box is able to support at most $x_i$ other boxes on top of it. If Farmer John wants to stack his boxes so that each box is directly on top of at most one other box, what is the minimum number of stacks that he'll need? [Adapted from \href{http://codeforces.com/problemset/problem/388/A}{Codeforces 388A}.]
\end{typewriter}

We first observe that if a stronger box is on top of a weaker box, then we can swap the two boxes and still have a valid stacking. Therefore, an optimal solution exists where no box is stronger than any box below it. This allows us to sort the boxes in increasing order of strength and construct an optimal solution by inserting stronger boxes below piles of weaker boxes.

Initially, we start with a single empty stack. We then add boxes one-by-one as follows: If we can, we insert our new box below some stack that it can successfully support. Otherwise, if no such stack exists, we create a new stack containing only the new box. It's possible to check all existing stacks in $O(n)$; therefore this algorithm runs in $O(n^2)$. (In fact, binary searching allows us to do the checking step in $O(\log n)$.) To prove correctness, we can again compare our construction to an arbitrary optimal construction. Finishing the argument is left as an exercise for the reader.

The subtle part about this problem is the order in which we process the boxes. Processing boxes from weakest to strongest offers an elegant greedy solution, while processing from strongest to weakest offers no simple approach. (At least I haven't found one yet.) Again, we see that ordering is important. Stepping back a bit, this problem also isn't one that immediately suggests a greedy approach. Since greedy solutions can often be unexpected, it's always worthwhile to take a moment to consider various orderings and ``na\"ive'' approaches to see if any work. Telltale hints of a greedy approach are observations about order and monotonicity---for example ``we can always have stronger boxes below weaker ones.''

\subsection{Snack Time}

Here's the \href{http://codeforces.com}{link} to the greedy problem set. Feast away!

\chapter*{Interlude B}
\addcontentsline{toc}{chapter}{Interlude B}
% \renewcommand\thesection{\Alph{chapter}.\arabic{section}}
% \setcounter{section}{0}

\section{Prefix Sums}

Suppose we were given an array of numbers of length $N$ and were asked to compute a series of $Q$ queries each asking for the sum of the numbers in the contiguous 
subsequence of numbers indexed from $i$ to $j$. 

\begin{center}
{
\begin{tikzpicture}[
  thick,
  myrect/.style={
    draw,
    fill=myseagreen,
    rectangle split,
    rectangle split horizontal,
    rectangle split parts=#1,
    rectangle split part align=left,
    text width=4ex,
    text centered
    }
]

\node[myrect=16]
  (array)
  {
  					\strut 2
  \nodepart{two}	\strut 4
  \nodepart{three}	\strut 7
  \nodepart{four}	\strut $-5$
  \nodepart{five}	\strut 3
  \nodepart{six}	\strut 6
  \nodepart{seven}	\strut $-3$
  \nodepart{eight}	\strut 1
  \nodepart{nine}	\strut $-2$
  \nodepart{ten}	\strut $-4$
  \nodepart{eleven}	\strut $-6$
  \nodepart{twelve}	\strut 2
  \nodepart{thirteen}	\strut 8
  \nodepart{fourteen}	\strut 6
  \nodepart{fifteen}	\strut 0
  \nodepart{sixteen}	\strut $-7$
  };
\foreach \Valor [count=\Valori from 1] in {one ,two ,three , four , five , six , seven , eight , nine , ten , eleven , twelve , thirteen , fourteen , fifteen , sixteen }
  \node[below] at (array.\Valor south) {\Valori};

\end{tikzpicture}
}
\end{center}

For example, the query on $[4,14]$ would sum 11 numbers and return the answer 6.
The na\"{i}ve solution would simply resum the numbers from position $i$ to position $j$ each time:

\begin{mylstlisting}
int arr[MAXN];
int query(int i, int j) {
	int sum = 0;
	for(int k = i; k <= j; ++k)
		sum += arr[k];
	return sum;
}
\end{mylstlisting}

In the worst case, this for loop is $O(N)$, since each query could loop over the majority of the size $N$ array.
This solution is fine if we only have a constant number of queries. But what if $Q$ is large? Notice that in case our na\"{i}ve solution is doing a lot of repeated work.
If we were to query $[3,12]$, for example, we would sum $[4,12]$ all over again, three fourths of the whole array! 

How can we eliminate this excess work? Our intuition from this specific example
would tell us to take our answer for $[4,14]$, subtract out anything that is not in the overlap ($[13,14]$),
and add any part of the new query that we haven't yet included ($[3,3]$). In this way we avoid summing over $[4,12]$ twice. But we quickly see that this solution is 
unnecessarily complicated and does not always speed up our solution: the ranges $[1,8]$ and $[9,16]$ are both half the array size but have no overlap whatsoever, so each
query could still be $O(N)$.

We find our answer using something called \textbf{prefix sums}. Prefix sums are an example of precomputation: performing some computation on our data before we read in any
queries to make the subsequent series of queries faster. Another example of precomputation you may have encountered before is sorting prior to a binary search, where we might
be queried how many numbers in a list are greater than $x$ for some number $x$. Na\"{i}vely checking each element in the list would result in a linear $O(N)$ time solution
per query, while sorting beforehand costs $O(N \log N)$ but results in each query costing $O(\log N)$.

In this case, we notice that if we knew the answers to the queries $[1,j]$ and $[1,i-1]$, we could simply subtract the two to get the answer to the query for $[i,j]$.
Consider the following array, where the number stored in index $i$ is the answer to the query $[1,i]$.

\begin{center}
\begin{tikzpicture}[
  thick,
  myrect/.style={
    draw,
    fill=myseagreen,
    rectangle split,
    rectangle split horizontal,
    rectangle split parts=#1,
    rectangle split part align=left,
    text width=4ex,
    text centered
    }
]

\node[myrect=17]
  (array)
  {
  					\strut 0
  \nodepart{two}	\strut 2
  \nodepart{three}	\strut 6
  \nodepart{four}	\strut 13
  \nodepart{five}	\strut 8
  \nodepart{six}	\strut 11
  \nodepart{seven}	\strut 17
  \nodepart{eight}	\strut 14
  \nodepart{nine}	\strut 15
  \nodepart{ten}	\strut 13
  \nodepart{eleven}	\strut 9
  \nodepart{twelve}	\strut 3
  \nodepart{thirteen}	\strut 5
  \nodepart{fourteen}	\strut 13
  \nodepart{fifteen}	\strut 19
  \nodepart{sixteen}	\strut 19
  \nodepart{seventeen}	\strut 12
  };
\foreach \Valor [count=\Valori from 0] in {one ,two ,three , four , five , six , seven , eight , nine , ten , eleven , twelve , thirteen , fourteen , fifteen , sixteen , seventeen }
  \node[below] at (array.\Valor south) {\Valori};

\end{tikzpicture}

\end{center}

Then the query for $[4,14]$ would simply subtract 13 from 19 to get 6. In general, once we have constructed this array, each query takes $O(1)$, since we simply look up
two numbers in the array and subtract them.

How might we construct this array in an efficient manner, though? We could use our na\"{i}ve solution $N$ times to get the $N$ sums that we need. This would then result
in an $O(N^2 + Q)$ overall solution. But the $N^2$ factor is very offputting. Here, we can make use of our observation that we are doing a lot of repeated work. If we know the
sum for $[1,i]$, then calculating the sum for $[1,i+1]$ is a simple matter of taking our previous answer and adding the $(i+1)$th term in the array:

\begin{mylstlisting}
int N;
int arr[MAXN];
int prefix[MAXN];
void preprocess() {
	prefix[0] = 0;
	for(int k = 1; k <= N; ++k)
		prefix[k] = prefix[k - 1] + arr[k];
}
int query(int i, int j) {
	return prefix[j] - prefix[i - 1];
}
\end{mylstlisting}

This solution is now $O(N + Q)$, much faster than the original $O(NQ)$ that we originally conceived!

So, at a high level, what happened here? We first noticed there was a lot of repetition in the work we were doing. We found a way to cut out the repetition by first 
finding the answer to a small (in this case, linear) number of nice cases. These nice cases have a certain structure that allow us to solve individual queries very quickly.
Finally, we exploited the same structure to compute those nice cases in a quick, neat fashion.

\section{Two Pointers}

