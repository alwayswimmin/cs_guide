\chapter{Big Ideas}

In this chapter we'll discuss some general ideas for solving problems. These include brute force,  depth-first search, greedy algorithms, binary search, and dynamic programming. We can think of them as the building blocks of more complex algorithms---each provides a very general approach to simplifying problems. Since the concepts we cover are independent of language, the algorithms presented will no longer be in the form of concrete Java or C++ code but rather in more abstract pseudocode.

\section{Brute Force}

Sometimes, the best way to approach a problem is to try everything. This idea of exhaustively searching all possibilities is called \emph{brute force}. For example, if we want to unlock a friend's iPhone, we could try all of the $10^4$ possible passcodes. As the name and this example suggest, brute force is often crude and inefficient. Usually we want to make some clever observations to make the problem more tractable. However, if the input size is small (check the number of operations against $10^8$) or if we just want to squeeze a few points out of a problem by solving only the small cases, brute force could be the way to go. And if you're stuck on a problem, thinking about a brute force is not a bad way to start. Simpler, slower algorithms can often inspire faster ones. Through the following problems, we'll show you how to brutally apply the idea of brute force.

\subsection{Square Root}

\begin{typewriter}
  Given an integer $n$, $1 \le n \le 10^{12}$, find the greatest integer less than or equal to $\sqrt{n}$ without using any library functions. (This means you can't call functions like Math.sqrt or Math.log.)
\end{typewriter}

At first, it is not obvious how we can compute square roots. However, we can always go simple. Set $i = 1$, and while $(i+1)^2\le n$, increment $i$. That is, we increment $i$ until increasing it further will cause $i$ to exceed $\sqrt n$. Since our answer $i$ is at most $\sqrt n \le 10^6$, our program runs in time. This is about the silliest approach we can use to calculate square roots, but hey, it works!

When implementing this algorithm, be careful about the size of $n$. The 32-bit \texttt{int} type in Java and C++ only holds values up to $2^{31} - 1 = 2,147,483,647$, which is exceeded by the maximum possible value of $n$. Thus we need to use a 64-bit integer type for our calculations: \texttt{long} in Java and \texttt{long long} in C++.

\subsection{Combination Lock}

\begin{typewriter}
  Farmer John purchases a combination lock to stop his cows from escaping their pasture and causing mischief! His lock has three circular dials, each with tick marks numbered $1$ through $N$ ($1\le N\le 100$), with $1$ and $N$ adjacent. There are two combinations that open the lock: one combination set by Farmer John and one "master" combination set by the locksmith. The lock has a small tolerance for error, however, so it will open if the numbers on each of the dials are at most two positions away from that of a valid combination. Given Farmer John's combination and the master combination, determine the number of distinct settings for the dials that will open the lock. 

  (For example, if Farmer John's combination is $(1,2,3)$ and the master combination is $(4,5,6)$, the lock will open if its dials are set to $(1,N,5)$ (since this is close to Farmer John's combination) or to $(2,4,8)$ (since this is close to the master combination). Note that $(1,5,6)$ would not open the lock, since it is not close enough to any single combination. Furthermore, order matters, so $(1,2,3)$ is distinct from $(3,2,1)$.) [Adapted from \href{http://usaco.org/index.php?page=viewproblem2&cpid=340}{USACO 2013, Combination Lock}.]
\end{typewriter}

Again, the simplest idea works. We can iterate over all possible settings of the lock, and for each setting, check if it matches either Farmer John's combination or the master combination. To do this iteration, we can use three nested \texttt{for} loops. The first loop through the values for the first dial, the second loop goes through the values for the second dial, and the third loop goes through the values for the third dial. Since there are three dials, the lock has at most $N^3 \le 10^6$ possible seetings. We can check if each dial matches in $O(1)$ time, hence our algorithm runs in less than a second.

In terms of implementation, \texttt{Combination Lock} is a great example of how a problem can decompose into two easier components that we can think about separately. The first component is to use nested loops to iterate through the possible settings, which we've described above. (Nested \texttt{for} loops like this show up often!) The second component is to check if a given setting is close to either of the given combinations. If we implement a function \texttt{is\_valid(a, b, c)} to do this, then the code becomes quite clean.

\subsection{Ski Course Design}

\begin{typewriter}
Farmer John has $N$ hills on his farm ($1\le N\le 1000$), each with an integer elevation in the range $0$ to $100$. In the winter, since there is abundant snow on these hills, he routinely operates a ski training camp. In order to evade taxes, Farmer John wants to add or subtract height from each of his hills so that the difference between the heights of his shortest and tallest hills is at most $17$ before this year's camp.

Suppose it costs $x^2$ dollars for Farmer John to change the height of a hill by $x$ units. Given the current heights of his hills, what is the minimum amount that Farmer John will need to pay? (Farmer John is only willing to change the height of each hill by an integer amount.) [Adapted from \href{http://usaco.org/index.php?page=viewproblem2&cpid=376}{USACO 2014, Ski Course Design}.]
\end{typewriter}

For \texttt{Ski Course Design}, we need to be a bit clever about how to implement our brute force. There are infinitely many ways we could change the heights of each hill, so it seems intractable to iterate over the possible heights for each hill separately. Instead, we look at the final range of the ski slope heights, which has length at most $17$. This final range has to fall within the interval $[0,100]$, hence there are less than $100$ possibilities. (The possible ranges are $[0, 17]$, $[1, 18]$, $\cdots$, $[83, 100]$.) Once we fix a range, we can calculate in $O(N)$ the minimum cost to make the height of each hill fall within that range. Thus if we let $M$ be the number of possible ranges ($M < 100$), we have an $O(MN)$ algorithm.

This problem shows that even when we brute force, we still have to think. Some approaches are better than others. In particular, we don't want to deal with cases that are irrelevant---for example, when the heights are not within a range of width $17$ or when Farmer John has not used the cheapest set of changes. We also don't want our possibilities to explode out of control, which would have happened had we adjusted the height of each hill separately with nested \texttt{for} loops or recursion. By iterating over a more restricted set of possibilities, we have created a brute force solution that runs significantly faster.

\subsection{Contest Practice}

\href{http://codeforces.com/group/iMPx86rZXm/contest/204642}{Here} is a collection of problems solvable through brute force. Try working through them on your own and applying the ideas you've seen so far. (May the brute force be with you.)

\section{Depth-First Search (DFS)}

Depth-first search is a recursive search technique that has a similar flavor to brute force. Here's an example. All of your friends, upon realizing that it only takes four nested \texttt{for} loops to iterate through all possible 4-digit iPhone passcodes, have decided to make their passcodes $n$ digits long. Since you don't know $n$, nested \texttt{for} loops will no longer do the trick. Instead, we can use a DFS to recursively generate all $n$-digit passcodes.

DFS works as follows: We check all passcodes starting with ``0'', then all passcodes starting with ``1'', then all passcodes starting with ``2'', and so on. To check all passcodes starting with ``0'', we check all passcodes starting with ``00'', then all passcodes starting with ``01'', then all passcodes starting with ``02'', and so on. To check all passcodes starting with ``00'', we have to check all passcodes starting with ``000'', then all passcodes starting with ... (Think about why DFS is \emph{depth-first}.)

In essence, we recursively generate all possible passcodes with a given prefix. We keep going down until we have to check a passcode starting with a string of length $n$, in which case that string is the passcode itself. Thus the first passcode we check is ``00$\cdots$0'' and the last passcode we check is ``99$\cdots$9''. Below is some pseudocode describing the algorithm. Make sure you understand how the recursion works!

\noindent \begin{minipage}{\textwidth}
  \begin{algorithmic}[1]
    \Function{DFS}{$depth$, $prefix$}
    \If{$depth = n$}
      \State \Call{check\_password\_on\_iPhone}{$prefix$}
      \State \Return
    \EndIf
    \For{$c$ from `0' to `9'}
      \State \Call{DFS}{$depth + 1$, $prefix + c$}
    \EndFor
    \EndFunction
  \end{algorithmic}
\end{minipage}

\subsection{Permutations}

\begin{typewriter}
  Given $n$ ($n \le 8$), print all permutations of the sequence $\{1, 2, \cdots, n\}$. (For $n=3$, we would have $(1, 2, 3)$, $(1, 3, 2)$, $(2, 1, 3)$, $(2, 3, 1)$, $(3, 1, 2)$, and $(3, 2, 1)$.)
\end{typewriter}

Like the passcode problem, we use DFS instead of nested \texttt{for} loops, since we don't know $n$. However, we have to be careful with implementation---we can use each number only once. Along with our current prefix, we have to keep track of the set of numbers that we've already used. This is best done with a boolean ``used'' array outside of the recursive function. Here's the pseudocode:

\noindent \begin{minipage}{\textwidth}
  \begin{algorithmic}[1]
    \State $used \gets \{false, false, \cdots, false\}$
    \Function{DFS}{$depth$, $prefix$}
    \If{$depth = n$}
      \State \Call{print}{$prefix$}
      \State \Return
    \EndIf
    \For{$i=1$ to $n$}
      \If{not $used[i]$}
        \State $used[i] \gets true$
        \State \Call{DFS}{$depth + 1$, $prefix + i$}
        \State $used[i] \gets false$
        \Comment We have to reset the $used[i]$ variable once we're done.
      \EndIf
    \EndFor
    \EndFunction
  \end{algorithmic}
\end{minipage}

\subsection{Basketball}

\begin{typewriter}
Two teams are competing in a game of basketball: the Exonians and the Smurfs. There are $n$ players on the Exonian team and $m$ players on the Smurf team, with $n + m \le 17$. Each player has an integer skill level $s$ between $1$ and $10^8$. Define the strength of a set of players as the sum of their individual skill levels. In order to ensure a fair game, the Exonians and Smurfs plan on choosing two equally strong starting lineups. In how many ways can the teams choose their lineups? (Two games are considered different if there exists a player who starts in one game, but not in the other.)
\end{typewriter}

We use a DFS to recursively generate all possible starting lineups. Each starting lineup can be represented by a sequence of $n + m$ 0's and 1's, where a player starts if and only if he/she is assigned a 1. We do this the same way we generate all passcodes of length $n + m$. Once we have a starting lineup, it is straightforward to check for fairness. (Is it also possible to keep track of the strength of each team as we DFS? Hint: Keep an extra variable similar to ``used'' in \texttt{Permutations}.)

\subsection{Problem Break}

Before moving on, try to implement the DFSes described above. You can test your code on the problem set \href{http://codeforces.com}{here}. How does the runtime of these algorithms grow with respect to $n$?

\subsection{Generalizing DFS}

Thus far, all of the DFS solutions that we've seen have involved sequences. However, we can also use DFS in a much more general setting. In the same spirit as brute force, if we want to enumerate or construct something and we have to make some decisions at each step, we can try all the options for each step and recurse. Essentially, we have a powerful, recursive method to check all the possibilities. The examples below show the flexibility of depth-first search---a huge class of problems can be solved in a brute force manner like this.

\subsection{Dungeon}

\begin{typewriter}
  Bessie is trying to escape from the dungeon of the meat-packing plant! The dungeon is represented by an $n$-by-$n$ grid where each of the grid cells is trapped and can only be stepped on once. Bessie is currently located in the lower-left corner and wants to make to the exit in the upper-right corner. How many paths can Bessie take to escape, assuming that she steps on no cell twice?
\end{typewriter}

\subsection{$n$ Queens Puzzle}

\begin{typewriter}
  Given $n$ ($4 \le n \le 8$), find an arrangement of $n$ queens on an $n$-by-$n$ chessboard so that no two queens attack each other. 
\end{typewriter}

\begin{typewriter}
\end{typewriter}
