\chapter{Implementation Tricks}

\subsection{Comparison}

We'll finish our section on sorting by taking a look at how Java and C++ sorts objects. We also provide a brief glimpse into using standard data structures that require a natural ordering on the objects within the container.

In Java, there is a built-in sort method, \texttt{Arrays.sort()}, in \texttt{java.util.Arrays}, which takes an array and sorts it. To sort an array \texttt{int array[]}, we call \texttt{Arrays.sort(array)}. Java always sorts least to greatest.

If we want to sort an array of our own objects, we have to first implement the \texttt{Comparable} interface. This is how Java orders a set of objects. Implementing \texttt{Comparable} requires support of the method \texttt{int compareTo(Object o)}. If \texttt{a.compareTo(b) < 0}, then \texttt{a} comes before \texttt{b}; if \texttt{a.compareTo(b) > 0}, then \texttt{a} comes after \texttt{b}; if \texttt{a.compareTo(b) == 0}, then \texttt{a} is considered equal to \texttt{b}. (\texttt{a.compareTo(b) == 0} should be equivalent to \texttt{a.equals(b)}.) For example, consider the class \texttt{MyPair}:

\begin{mylstlisting}
class MyPair implements Comparable {
	int x;
	int y;
    // sort by x-coordinate first
	public int compareTo(Object o) {
		MyPair c = (MyPair) o;
		if(x < c.x) return -1;
		if(x > c.x) return 1;
		// if x-coordinates equal, compare y
		if(y < c.x) return -1;
		if(y > c.x) return 1;
		return 0; // equal
	}
}
\end{mylstlisting}

If you know generics (Section \ref{sec:generics}), we can clean this up a bit. Note that no casting is required anymore.

\begin{mylstlisting}
class MyPair implements Comparable<MyPair> {
	int x;
	int y;
	public int compareTo(MyPair c) {
		if(x < c.x) return -1;
		if(x > c.x) return 1;
		if(y < c.x) return -1;
		if(y > c.x) return 1;
		return 0;
	}
}
\end{mylstlisting}

We can sort much more than just arrays. Java provides us many data structures we can use, much more than just arrays. In this section, I'll mention briefly the \texttt{ArrayList} for Java and the \texttt{vector} for C++. If you don't know what either of these are, you can just pretend like they are arrays for now; they are very similar and will be covered in detail in a later chapter. To sort these more complex data structures in Java, we use \texttt{Collections.sort()}.

Suppose we wanted to code a custom comparison operation for something that already exists. For example, suppose we wanted to sort an array of \texttt{ArrayList}s based on the sum of the elements in the \texttt{ArrayList}. We clearly don't want to code an entire \texttt{ArrayList} just because we want to sort it in a special way. One way we can easily resolve this is by using \texttt{extends}.

\begin{mylstlisting}
class MyArrayList extends ArrayList<Integer> implements Comparable<MyArrayList> {
	public int compareTo(MyArrayList c) {
		int sum = 0;
		for(int k = 0; k < size(); k++) {
			sum += get(k);
		}
		int csum = 0;
		for(int k = 0; k < c.size(); k++) {
			csum += c.get(k);
		}
		if(sum < csum) return -1;
		if(sum == csum) return 0;
		return 1;
	}
}
\end{mylstlisting}

Alternatively, we could use a \texttt{Comparator}. This is considered much better practice than the hack I presented earlier by coding a new class with \texttt{extends}. \texttt{Comparator}s are especially useful if we are to compare the same object different ways in the same program. A \texttt{Comparator} is an entirely separate class that takes two objects and determines which ought to come before the other in the ordering. Here is the \texttt{Comparator} implementation of the example I showed above.

\begin{mylstlisting}
class Cmp implements Comparator<ArrayList<Integer>> {
	public int compare(ArrayList<Integer> o1, ArrayList<Integer> o2) {
		int sum1 = 0;
		for(int k = 0; k < o1.size(); k++) {
			sum1 += o1.get(k);
		}
		int sum2 = 0;
		for(int k = 0; k < o2.size(); k++) {
			sum2 += o2.get(k);
		}
		if(sum1 < sum2) return -1;
		if(sum1 == sum2) return 0;
		return 1;
	}
}
\end{mylstlisting}

Now, when we sort, we'll need to pass our \texttt{Comparator} in addition to the data structure to be sorted. For example, we use the code \texttt{Arrays.sort(array, new Cmp())}. When we use a data structure that requires a custom comparison, we use a different constructor: \texttt{new TreeSet<ArrayList<Integer>>(new Cmp())}.

In C++, comparing objects is slightly different. We have three methods, and all can be used with \texttt{sort}, which is in \texttt{<algorithm>}, or with any other standard library function or data structure which requires an ordering. As in Java, sorting is done least to greatest.

The first method is to implement the method \texttt{operator<} in a \texttt{struct} or \texttt{class} we coded ourselves. Here's an example \texttt{struct my\_pair} which is equivalent to the above Java \texttt{class MyPair}.

\begin{mylstlisting}[language=C++]
struct my_pair {
	int x;
	int y;
	friend bool operator<(const my_pair& lhs, const my_pair& rhs) {
		return lhs.x < rhs.x || lhs.x == rhs.x && lhs.y < rhs.y;
    }
};
\end{mylstlisting}

Suppose we have an array \texttt{my\_pair array[]}. We can then call \texttt{sort(array, array + n)} to sort the first \texttt{n} elements of the array. If we have a different data structure, like \texttt{vector<my\_pair> v}, we can sort it using \texttt{sort(v.begin(), v.end())}.

The second method is to use a custom comparison function. This is very similar to what we would put in function contained in the Java \texttt{Comparator}, but is distinct from the \texttt{Comparator} itself as it is only a function and not a class. Suppose we wanted to sort pairs primarily increasing in the first element and secondarily decreasing in the second. It seems inefficient to code a new pair class when C++ provides one, so we can simply use a custom comparison function to get around this.

\begin{mylstlisting}[language=C++]
typedef std::pair<int, int> pii;
bool cmp(const pii& lhs, const pii& rhs) {
	return lhs.x < rhs.x || lhs.x == rhs.x && lhs.y > rhs.y;
}
\end{mylstlisting}

\texttt{cmp}, like \texttt{operator<}, should return \texttt{true} when \texttt{lhs} is considered to come before \texttt{rhs} in the ordering. To use \texttt{sort} here, we pass a third argument containing our custom comparison: \texttt{sort(array, array + n, cmp)}. Passing a custom comparison function to a standard library data structure is possible but unfortunately quite confusing and will not be covered here. This can be done more easily using the final method.

The final method uses a functor, which is an object that can also work as a function. This is done by implementing the function \texttt{operator()()}. In the case of sorting, the functor works somewhat like a Java \texttt{Comparator}.

\begin{mylstlisting}[language=C++]
typedef std::pair<int, int> pii;
struct cmp {
	bool operator()(const pii& lhs, const pii& rhs) {
		return lhs.x < rhs.x || lhs.x == rhs.x && lhs.y > rhs.y;
	}
};
\end{mylstlisting}

The functor method works just as the custom comparison function for \texttt{sort}: \texttt{sort(array, array + n, cmp)}. However, since it is now an object, it can be passed as a template argument to standard library containers much more easily than the comparison function can. For example, for the binary search tree implemented by C++, we can pass a custom comparison using \texttt{set<int, cmp> s}.

