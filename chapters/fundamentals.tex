\chapter{Fundamentals}

\section{Input and Output}

The first part of solving any programming contest problem is reading the input correctly. In this section, we'll briefly go over input and output in Java using \texttt{java.util.Scanner} and \texttt{java.io.PrintWriter}. There are two scenarios that you should be familiar with: \texttt{stdin}/\texttt{stdout} and file I/O. You encounter the former when you enter input and see output for a program run in the commandline. You encounter the latter when you have two files (for example \texttt{input.txt} and \texttt{output.txt}) that you read from and write to.

When using \texttt{stdin}/\texttt{stdout}, we read input from \texttt{System.in} using a Scanner and output our results using \texttt{System.out.println()}. \texttt{Scanner.nextInt()} and \texttt{Scanner.next()} read in integers and strings, respectively. \texttt{System.out.println()} prints its argument and adds a newline at the end. (If we don't want the newline, we can use \texttt{System.out.print()}.) Here's an example of a main method that takes two integers and outputs their sum:

\begin{mylstlisting}
public static void main(String args[]){
  // hint: you should write "import java.util.*;" at the top of your code.
  Scanner sc = new Scanner(System.in);
  int x = sc.nextInt();
  int y = sc.nextInt();
  System.out.println(x + y);
}
\end{mylstlisting}

File I/O is a touch more complicated. For our Scanner, we have to make a new File object and use it in the constructor. We do the same for our PrintWriter. However, PrintWriter also comes with a couple more usage notes. First, we should include \texttt{throws IOException} after our main method, since Java requires that we acknowledge the possibility of an IOException. (We bravely assume that our file open will succeed!) After we finish printing, we must also close the PrintWriter to make sure that everything gets written. Here's a snippet showing how Scanner and PrintWriter work with files: 

\begin{mylstlisting}
public static void main(String args[]) throws IOException {
  // hint: for file I/O, you should also have "import java.io.*;"
  Scanner sc = new Scanner(new File("input.txt"));
  int x = sc.nextInt();
  int y = sc.nextInt();
  PrintWriter pw = new PrintWriter(new File("output.txt"));
  pw.println(x + y);
  pw.close();
}
\end{mylstlisting}

Although more efficient methods of I/O exist, such as BufferedReader and BufferedWriter, what's covered here should be sufficient for now. (It's possible to read in $10^5$ integers with Scanner in a fraction of a second.)

Now that we can read and write, let's move on to one of the most basic subroutines in computer science: sorting. 

\section{Comparable}

We'll start by taking a look at how Java sorts objects. In \texttt{java.util.Arrays}, there is a built-in sort method, \texttt{Arrays.sort()}, which takes an array and sorts it. If we want to sort an array of our own objects, we have to first implement the \texttt{Comparable} interface. This is how Java orders a set of objects. Implementing \texttt{Comparable} requires support of the method \texttt{int compareTo(Object o)}. If \texttt{a.compareTo(b) < 0}, then \texttt{a} comes before \texttt{b}; if \texttt{a.compareTo(b) > 0}, then \texttt{a} comes after \texttt{b}; if \texttt{a.compareTo(b) == 0}, then \texttt{a} is considered equal to \texttt{b}. (\texttt{a.compareTo(b) == 0} should be equivalent to \texttt{a.equals(b)}.) For example, consider the following \texttt{class MyPair}:

\begin{mylstlisting}
class MyPair implements Comparable {
	int x;
	int y;
    // sort by x-coordinate first
	public int compareTo(Object o) {
		MyPair c = (MyPair) o;
		if(x < c.x) return -1;
		if(x > c.x) return 1;
		// if x-coordinates equal, compare y
		if(y < c.x) return -1;
		if(y > c.x) return 1;
		return 0; // equal
	}
}
\end{mylstlisting}

We can clean this up a bit using generics:

\begin{mylstlisting}
class MyPair implements Comparable<MyPair> {
	int x;
	int y;
	public int compareTo(MyPair c) {
		if(x < c.x) return -1;
		if(x > c.x) return 1;
		if(y < c.x) return -1;
		if(y > c.x) return 1;
		return 0;
	}
}
\end{mylstlisting}

Note that no casting is required anymore.

\section{Sorts}

This section covers three of the more practical sorting algorithms.

\subsection{Insertion Sort}

Insertion sort builds up the sorted list by inserting new elements one at a time. Inserting an element into a sorted list takes time proportional to the length of the list, so the runtime of this algorithm is $O(n^2)$. This algorithm, despite being slower than merge sort and quicksort, is still useful because of its efficiency on small inputs. Many implementations of merge sort and quicksort actually use insertion sort once the problem size gets small.

(The $O(n^2)$, called big-O notation, means our algorithm does at most $cn^2$ operations, for some constant $c$.)

\subsection{Merge Sort}

The idea behind merge sort is the following observation: If we are given two sorted lists of length $n/2$, we only need $n$ comparisons to merge them into one sorted list of length $n$. So we can divide and conquer. We do this by cutting the array in half, sorting each half recursively, and merging the two halves back together. Because our recursion goes $\log_2 n$ levels deep, this algorithm runs in $O(n \log n)$.

\subsection{Quicksort}

Quicksort also uses a divide and conquer strategy to achieve an $O(n \log n)$ amortized bound. We take a random element from the array, move anything less than it to the left of the array, anything greater than it to the right of the array, and recurse on the two halves that we just created.

This was previously implemented by \texttt{Arrays.sort()} and \texttt{Collections.sort()}, but Java now uses dual-pivot quicksort and timsort.

\section{Binary Search}

Before we begin studying data structures, it is necessary to first understand some search techniques. Suppose we're given a list and we want to check if it contains some element. We can do this directly by looking at every element of the list in a \emph{linear search}.

One way we can make this faster is by imposing an ordering on the list. Then inserting elements becomes slower because we can no longer just pop the element to the the end of the list. However, if we are asked to find an element, we know roughly where to search because we can compare our element to other elements in the list. In fact, we only need to look at $O(\log n)$ elements to check if the queried element exists.

Consider a sorted array that supports $O(1)$ access to any element. We can compare our queried element to the middle element of the array. Depending on whether our element is larger, smaller, or equal to the middle element, we'll be able to eliminate at least half of the array. Thus finding an element in a sorted array with \emph{binary search} is an $O(\log{n})$ operation. This can be done either iteratively or recursively.

