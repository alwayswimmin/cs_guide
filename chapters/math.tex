\chapter{Math}

Algorithms here exhibit a different flavor than the graph theory, string, or geometry algorithms from earlier. This chapter is placed towards the end because material here doesn't really fit in any other section or the USACO canon, \textit{not} because this material is particularly difficult.

\section{Big Integer}

A \textit{big int} is for times when you need integers greater than $2^{63}-1$, when \texttt{int} and \texttt{long} just aren't large enough. The way it works is it stores a number as an array of ints. Each value in the array represents a digit in some very large base. Addition and subtraction can be done in the standard way. Generally multiplying two big ints is not necessary on contests, but it can be sped up using multiplication algorithms like Karatsuba.

\texttt{BigInteger} is in \texttt{java.math}, but C++ does not have an implementation.

\section{Number Theory}

In this section we explore ways for computers to quickly compute some useful mathematical quantities.

\section{Combinatorial Games}

\url{https://activities.tjhsst.edu/sct/lectures/1314/impartial_games_12_06_13.pdf}

\section{Karatsuba}

\section{Matrices}

\section{Fast Fourier Transform}

Hi, I'll try to write this.