\chapter{Math}

Algorithms here exhibit a different flavor than the graph theory, string, or geometry algorithms from earlier. This chapter is placed towards the end because material here doesn't really fit in any other section or the USACO canon, \textit{not} because this material is particularly difficult.

\section{Number Theory}

This section will contain much information about primes in many different ways. Indeed, prime numbers are the building blocks of number theory!

\subsection{Random Prime Numbers Bounds}

The point of this section is mostly to make the complexity analysis of the later sections easier. Therefore, I will list some useful theorems and bounds without proof. The proofs can be found in any standard number theory textbook or online. They are extremely insignificant in the context of the goals of this section.

\textbf{Prime Number Theorem}: The number of prime numbers less than $n$ is $O(\frac{n}{\log n})$ for any positive integer $n.$

\emph{Corollary}: The $n$-th prime number, $p_n$, is $O(n \log n).$

\textbf{Prime Number Harmonic Sum}: $\sum_{p \le n} \frac{1}{p} = O(\log \log n).$

\subsection{Prime Number Testing}

Let's say if you want to test is a number $n$ is prime. The most obvious way is to test all integers from $2$ to $n-1$, and see if they divide $n$. This takes $O(n)$ time. We can do better though! Note that if $n$ has a nontrivial factor, then it has one that is less than $\sqrt{n}.$ Indeed, if $n = pq,$ (as a nontrivial factorization) if both $p, q > \sqrt{n}$, then $pq > \sqrt{n}^2 = n$, a contradiction. So we only need to check whether $n$ has factors between $2$ and $\sqrt{n}.$ This takes $O(\sqrt{n})$ time. If you have precomputed a list of prime numbers (up to say $\sqrt{n}$), ten you only need to check whether $n$ has any prime factors in the same $2$ to $\sqrt{n}.$ This would give runtime $O(\frac{\sqrt{n}}{\log n})$ from the \emph{Prime Number Theorem} in section 10.1.1, a small but nontrivial improvement.

\subsection{Sieve of Eratosthenes}

If you want to compute all primes in the range $1$ to $n$. Runs in $O(n \log \log n)$. In other words, super fast.

\subsection{Prime Factorization}

How can we prime factorize a number? That is, given a positive integer $n$, how can we write it in the form $p_1 \cdot p_2 \cdot \dots \cdot p_k$, where the $p_i$ are all primes? Form example, $30 = 2 \cdot 3 \cdot 5$ and $24 = 2 \cdot 2 \cdot 2 \cdot 3$. The fact that every number can be uniquely factorized is called the \emph{Fundamental Theorem of Arithmetic}. Let's now figure out how to prime factorize every number efficiently. Using our intuition from above, we should expect that checking up to around $\sqrt{n}$ suffices. This is indeed true. The code below demonstrates:

\subsection{GCD and the Euclidean Algorithm}

I'll start once again with a theorem in number theory: if positive integers $a, b$ satisfy $\gcd(a, b) = 1$, then there exist integers $x, y$ such that $ax+by=1.$ This is called \emph{Bezout's Lemma.}

Similarly, if $\gcd(a, b) = g$, then there are integers $x, y$ such that $ax+by = g.$

%example gcd code here... and also some description of Euclidean Algorithm

\subsection{Fermat's Little Theorem}

Fermat's Little Theorem states that $a^p \equiv a \pmod{p}$ for any positive integer $a.$ Equivalently, we can say that $a^{p-1} \equiv 1 \pmod{p}.$

\subsection{Modular Inverses}

If $p$ is a prime, and $x$ satisfies $\gcd(x, p) = 1$, then there exists a positive integer $k$ such that $kx \equiv 1 \pmod{p}.$ $k$ is called the \emph{modular inverse} of $x \pmod{p},$ often just denoted as $1/x$ or $x^{-1}.$ Indeed, $1/x \cdot x = x^{-1} \cdot x = 1,$ so this notation is natural. The cool thing about modular inverses is that you can see them as division modulo $p.$ Since $x$ is nonzero $\pmod{p},$ it makes sense that you should be able to divide by $x.$ This is the main application of modular inverses. But how to find one quickly? There are two common algorithms, both of which run in $O(\log p)$ time. You can do this by running the Euclidean Algorithm described above: find positive integers $j, k$ such that $pj + kx = 1.$ Then obviously $kx \equiv 1 \pmod{p}.$ The other way is to notice that $x^{p-1} \equiv 1 \pmod{p}$ by Fermat's Little Theorem, so $x^{p-2} \equiv x^{-1} \pmod{p}.$ Now compute the left hand side of the previous expression using binary exponentiation.

Exercises related: Do $O(n \log p)$ precomputation to compute any binomial coefficient of the form $\binom{m}{k} \pmod{p}$ where $0 \le m, k \le n$ in $O(1)$ time.

\section{Combinatorial Games}

\url{https://activities.tjhsst.edu/sct/lectures/1314/impartial_games_12_06_13.pdf}

\section{Karatsuba}

\section{Matrices}

\section{Fast Fourier Transform}

\subsubsection{Introduction}

The \emph{Fast Fourier Transform} (FFT) is a technique used to multiply two polynomials of degree $n$ in $O(n \cdot \log n)$ time. At a high level, what this algorithm is doing is something called \emph{polynomial interpolation}. This refers to the fact that if we know the value of a polynomial $P$ of degree $n$ at $n+1$ points, then we can uniquely determine the polynomial $P.$ The proof of this statement is simple, and involves the Lagrange Interpolation Formula for one direction and a simple root counting argument to prove uniqueness. I won't go into detail here because the proof isn't important, but it is useful to know some motivation behind the algorithm.

\subsubsection{Algorithm Outline}

Let's say we want to multiply 2 polynomials $A(x), B(x)$, both of degree $n$. Let $C(x) = A(x)B(x).$ The algorithm will proceed in 3 steps. Choose an integer $m > 2n,$ and choose $m$ numbers $x_0, x_1, \dots, x_{m-1}.$ I'll clarify what to choose $m$ and the $x_i$ as later. Just keep in mind that we can choose these values to be anything we want.

\begin{enumerate}

\item Evaluate $A(x_0), A(x_1), \dots, A(x_{m-1}).$ Similarly, evaluate $B(x_0), B(x_1), \dots, B(x_{m-1}).$

\item Evaluate $C(x_i) = A(x_i)B(x_i)$ for $0 \le i \le m-1.$

\item Interpolate the coefficients of $C(x)$ given the values $C(x_0), C(x_1), \dots, C(x_{m-1}).$

\end{enumerate}

The last step explains why we need $m > 2n.$ The degree of $C(x)$ is $2n$, so we need at least $2n+1$ points to determine $C(x)$ uniquely.

You should be skeptical that the above approach does any better than $O(n^2).$ In particular, step 1 seems like it should take $O(n^2).$ It turns out that if we choose the values $x_0, x_1, \dots, x_{m-1}$ properly, then we can do much better. Let me now describe what to choose $m$ to be, and what to choose $x_0, x_1, \dots, x_{m-1}$ to be.

\subsubsection{Roots of Unity}

Before telling you precisely, here's some food for thought. Let's take then example polynomial $A(x) = 1+2x+3x^2+4x^3+5x^4+6x^5+7x^6+8x^7.$ Let's split this into 2 groups: the even degree coefficients and odd degree coefficients. Let's call these two groups $A_{even}$ and $A_{odd}.$ Define $A_{even}(x) = 1+3x+5x^2+7x^3, A_{odd}(x) = 2+4x+6x^2+8x^7.$ Then clearly \[ A(x) = A_{even}(x^2) + x \cdot A_{odd}(x^2) \]

Notice that the $x^2$ in the formula makes it extremely easy to compute $A(-x)$ given $A(x).$ It's only one sign flip off! Therefore, it would be cool if our $x_i$ above were symmetric with respect to 0, i.e. if we want to compute $x$, we also should want to compute $-x$. An example set like this is $\{1, 2, 3, 4, -1, -2, -3, -4 \}.$ So if we wanted to compute $A(x)$ at these values, we would need to compute the values of $A_{even}, A_{odd}$ at their squares, that is $\{1, 4, 9, 16 \}.$ But this set is no longer symmetric! So this is not what we want exactly, since that means that we can't just recursively compute $A_{even}$ and $A_{odd}.$ So let's say that the set we want to compute $A_{even}, A_{odd}$ on is something like $\{1, 4, -1, -4 \}.$ Then the values that we get for $A(x)$ are $\{1, -1, 2, -2, i, -i, 2i, -2i \}.$ Complex numbers! This is closer to what we want, but still not precisely. The selection of the $x_i$ explained below makes everything work out perfectly.

So what should we choose $m$ and the $x_i$ to be? I'll tell you now and then explain why this selection works so well: choose $m$ to be a power of $2$ that is larger than $2n$, and choose $x_0, x_1, \dots, x_{m-1}$ to be the $m$\emph{-th roots of unity}. The $m$-th roots of unity are complex numbers that satisfy the equation $x^m = 1.$ They are of the form $\cos\left(\frac{2k\pi i}{m}\right) + i \cdot \sin\left(\frac{2k\pi i}{m} \right)$ for any integer $k$ from $0$ to $m-1.$

Let $\omega$ be a \emph{primitive} root of unity, i.e. the smallest $m'$ such that $\omega^{m'} = 1$ is $m' = m.$ We can without loss of generality set $\omega = \cos\left(\frac{2\pi i}{m}\right) + i \cdot \sin\left(\frac{2\pi i}{m} \right).$ One can easily check the the remaining roots of unity are $\omega^2, \omega^3, \dots, \omega^m.$ From now on, let's just set $x_i = \omega^i$ for all $0 \le i \le m-1.$ Note that $x_0 = 1.$ Also set $m = 2^k > 2n.$ Now I can proceed to describing why this algorithm works.

\subsubsection{Algorithm Specifics: Step 1}

In this section I'll implement (using pseudocode) a function 

vector<double> fft(vector<int> $A$, k, $\omega$). This means that this will return a vector<double> of length $2^k$ containing the values $A(\omega^0), A(\omega^1), A(\omega^2), \dots, A(\omega^{2^k-1}).$ Remember that $\omega^{2^k} = 1.$ The vector<int> $A$ stores the coefficients of $A(x)$. The $x^i$ coefficient of $A$ would be stored as $A[i].$

Here's an implementation. The brackets $\{ \}$ will be shorthand for containing a vector.

\begin{algorithm}[H]
\caption{Fast Fourier Transform}
%\label{}
\begin{algorithmic}
\Function{FFT}{vector<int> $A, k, \omega$} \Comment $\omega^{2^k} = 1$
	\State $A$.resize($2^k$)
	\State vector<double> $f, f_{even}, f_{odd}$
	\State $f$.resize($2^k$)
	
	\If {$k = 0$}
		\State \Return $ \{A[0] \}$ \Comment a vector of length 1
    \EndIf
    \State $A_{even}$ = $ \{A[0], A[2], \dots, A[2^k-2] \}$
    \State $A_{odd}$ = $ \{A[1], A[3], \dots, A[2^k-1] \}$
    \State $f_{even} = $ \Call{FFT}{$A_{even}, k-1, \omega^2$}
    \State $f_{odd} = $ \Call{FFT}{$A_{odd}, k-1, \omega^2$}
    
    \ForAll {$i = 0, i < 2^{k-1}, i++$}
    	\State $f[i] = f_{even}[i] + \omega^i \cdot f_{odd}[i]$
	\State $f[i+2^{k-1}] = f_{even}[i] - \omega^i \cdot f_{odd}[i]$

	\EndFor

\Return f

\EndFunction

\end{algorithmic}
\end{algorithm}

So why does this algorithm work? Note that the algorithm proceeds recursively, calling itself twice (once for even, once for odd). $A_{even}(x)$ is the polynomial $A[0] + A[2]x + A[4]x^2 + \dots + A[2^k-2]x^{2^{k-1}-1}.$ Therefore, what the recursive call is doing is computing the values of $A_{even}$ at the values $x = \omega^0, \omega^2, \omega^4, \dots, \omega^{2^k-2}.$ This is equivalent to computing the values of the polynomial $B_{even}(x) = A[0] + A[2]x^2 + A[4]x^4 + \dots + A[2^k-2]x^{2^{k}-2}$ at the values $x = \omega^0, \omega^1, \omega^2, \dots, \omega^{2^{k-1}-1}.$ The recursive call for $A_{odd}(x) = A[1] + A[3]x + A[5]x^2 + \dots + A[2^k-1]x^{2^{k-1}-1}$ behaves in a similar way. Similarly, define $B_{odd}(x) = A[1] + A[3]x^2 + A[5]x^4 + \dots + A[2^k-1]x^{2^{k}-2}.$

The key is to note that $A(x) = B_{even}(x) + x \cdot B_{odd}(x).$ Since we know the values of $B_{even}(x), B_{odd}(x)$ for $x = \omega^0, \omega^1, \dots, \omega^{2^{k-1}-1},$ we also can compute the values of $A(x)$ for these values of $x$. But what about the remaining values? This requires the observation that $\omega^{i + 2^{k-1}} = -\omega^i.$ Since $B_{even}(x), B_{odd}(x)$ only have even powers, $B_{even}(\omega^i) = B_{even}(-\omega^i) = B_{even}(\omega^{i + 2^{k-1}}).$ The same equality also holds for $B_{odd}(x).$ Using this observation along with the equation $A(x) = B_{even}(x) + x \cdot B_{odd}(x),$ we can see now why the two equations in the for loop of the code above hold. And that's how we do Step 1!

Let's analyze the runtime of this function FFT. Let $T(2^k)$ denote the time needed to run FFT on an array of length $2^k.$ Then $T(2^k) = 2T(2^{k-1}) + O(2^k) \implies T(2^k) = k \cdot 2^k,$ by the Master Theorem. Since $2^k$ is $O(n)$ by the above discussion, this steps runs in $O(n \cdot \log n)$ time.

\subsubsection{Algorithm Specifics: Step 2}

I'll give you a rest after a hard theoretical previous section. You do this step by looping through the values $A(x_0), A(x_1), \dots, A(x_{2^k-1}), B(x_0), B(x_1), \dots, B(x_{2^k-1})$ and directly multiplying $C(x_i) = A(x_i)B(x_i).$ This step runs in $O(n)$ time.

\subsubsection{Algorithm Specifics: Step 3}

This may seem like something completely new, but actually most of the work is already done. I'll explain. In fact I claim that writing the code $C = \text{fft}(C, k, \omega^{-1}),$ and then dividing all elements of $C$ by $2^k$ finishes. Step 3 is very easy once you have Step 1!

To see this, let's take a look at how step 1 works. We'll consider the coefficients of $A$ as a vector (not a C++ vector, but an actual math vector -- matrices!).

\begin{align*}
\begin{pmatrix}
A(\omega^0) \\ 
A(\omega^1) \\ 
A(\omega^2) \\ 
A(\omega^3) \\ 
\vdots \\
A(\omega^{2^k - 2}) \\
A(\omega^{2^k - 1})
\end{pmatrix}
&=
\begin{pmatrix}
\omega^0 & \omega^0 & \omega^0 & \omega^0 & \cdots & \omega^0 & \omega^0 \\ 
\omega^0 & \omega^1 & \omega^2 & \omega^3 & \cdots & \omega^{2^k-2} & \omega^{2^k-1} \\ 
\omega^0 & \omega^2 & \omega^4 & \omega^6 & \cdots & \omega^{2(2^k-2)} & \omega^{2(2^k-1)} \\ 
\omega^0 & \omega^3 & \omega^6 & \omega^9 & \cdots & \omega^{3(2^k-2)} & \omega^{3(2^k-1)} \\ 
\vdots   & \vdots   & \vdots   & \vdots   & \ddots & \vdots            & \vdots            \\
\omega^0 & \omega^{2^k - 2} & \omega^{2(2^k - 2)} & \omega^{3(2^k - 2)} & \cdots & \omega^{(2^k - 2)(2^k-2)} & \omega^{(2^k-1)(2^k - 2)} \\ 
\omega^0 & \omega^{2^k - 1} & \omega^{2(2^k - 1)} & \omega^{3(2^k - 1)} & \cdots & \omega^{(2^k - 2)(2^k-1)} & \omega^{(2^k-1)(2^k - 1)}
\end{pmatrix}
\begin{pmatrix}
A[0] \\ 
A[1] \\ 
A[2] \\ 
A[3] \\
\vdots \\
A[2^k - 2] \\
A[2^k - 1]
\end{pmatrix}
\end{align*}

Now suppose we perform the FFT with $\omega^{-1}$. This is equivalent by multiplying by a similar matrix.

\eject \pdfpagewidth=13in \pdfpageheight=11in

\begin{align*}
&\phantom{= }
\begin{pmatrix}
\omega^0 & \omega^0 & \omega^0 & \omega^0 & \cdots & \omega^0 & \omega^0 \\ 
\omega^0 & \omega^{-1} & \omega^{-2} & \omega^{-3} & \cdots & \omega^{-(2^k-2)} & \omega^{-(2^k-1)} \\ 
\omega^0 & \omega^{-2} & \omega^{-4} & \omega^{-6} & \cdots & \omega^{-2(2^k-2)} & \omega^{-2(2^k-1)} \\ 
\omega^0 & \omega^{-3} & \omega^{-6} & \omega^{-9} & \cdots & \omega^{-3(2^k-2)} & \omega^{-3(2^k-1)} \\ 
\vdots   & \vdots   & \vdots   & \vdots   & \ddots & \vdots            & \vdots            \\
\omega^0 & \omega^{-(2^k - 2)} & \omega^{-2(2^k - 2)} & \omega^{-3(2^k - 2)} & \cdots & \omega^{-(2^k - 2)(2^k-2)} & \omega^{-(2^k-1)(2^k - 2)} \\ 
\omega^0 & \omega^{-(2^k - 1)} & \omega^{-2(2^k - 1)} & \omega^{-3(2^k - 1)} & \cdots & \omega^{-(2^k - 2)(2^k-1)} & \omega^{-(2^k-1)(2^k - 1)}
\end{pmatrix}
\begin{pmatrix}
A(\omega^0) \\ 
A(\omega^1) \\ 
A(\omega^2) \\ 
A(\omega^3) \\ 
\vdots \\
A(\omega^{2^k - 2}) \\
A(\omega^{2^k - 1})
\end{pmatrix}
\\
&=
\begin{pmatrix}
\omega^0 & \omega^0 & \omega^0 & \omega^0 & \cdots & \omega^0 & \omega^0 \\ 
\omega^0 & \omega^{-1} & \omega^{-2} & \omega^{-3} & \cdots & \omega^{-(2^k-2)} & \omega^{-(2^k-1)} \\ 
\omega^0 & \omega^{-2} & \omega^{-4} & \omega^{-6} & \cdots & \omega^{-2(2^k-2)} & \omega^{-2(2^k-1)} \\ 
\omega^0 & \omega^{-3} & \omega^{-6} & \omega^{-9} & \cdots & \omega^{-3(2^k-2)} & \omega^{-3(2^k-1)} \\ 
\vdots   & \vdots   & \vdots   & \vdots   & \ddots & \vdots            & \vdots            \\
\omega^0 & \omega^{-(2^k - 2)} & \omega^{-2(2^k - 2)} & \omega^{-3(2^k - 2)} & \cdots & \omega^{-(2^k - 2)(2^k-2)} & \omega^{-(2^k-1)(2^k - 2)} \\ 
\omega^0 & \omega^{-(2^k - 1)} & \omega^{-2(2^k - 1)} & \omega^{-3(2^k - 1)} & \cdots & \omega^{-(2^k - 2)(2^k-1)} & \omega^{-(2^k-1)(2^k - 1)}
\end{pmatrix}
\begin{pmatrix}
\omega^0 & \omega^0 & \omega^0 & \omega^0 & \cdots & \omega^0 & \omega^0 \\ 
\omega^0 & \omega^1 & \omega^2 & \omega^3 & \cdots & \omega^{2^k-2} & \omega^{2^k-1} \\ 
\omega^0 & \omega^2 & \omega^4 & \omega^6 & \cdots & \omega^{2(2^k-2)} & \omega^{2(2^k-1)} \\ 
\omega^0 & \omega^3 & \omega^6 & \omega^9 & \cdots & \omega^{3(2^k-2)} & \omega^{3(2^k-1)} \\ 
\vdots   & \vdots   & \vdots   & \vdots   & \ddots & \vdots            & \vdots            \\
\omega^0 & \omega^{2^k - 2} & \omega^{2(2^k - 2)} & \omega^{3(2^k - 2)} & \cdots & \omega^{(2^k - 2)(2^k-2)} & \omega^{(2^k-1)(2^k - 2)} \\ 
\omega^0 & \omega^{2^k - 1} & \omega^{2(2^k - 1)} & \omega^{3(2^k - 1)} & \cdots & \omega^{(2^k - 2)(2^k-1)} & \omega^{(2^k-1)(2^k - 1)}
\end{pmatrix}
\begin{pmatrix}
A[0] \\ 
A[1] \\ 
A[2] \\ 
A[3] \\
\vdots \\
A[2^k - 2] \\
A[2^k - 1]
\end{pmatrix}
\\
&=
\begin{pmatrix}
2^k & 0 & 0 & 0 & \cdots & 0 & 0 \\ 
0 & 2^k & 0 & 0 & \cdots & 0 & 0 \\ 
0 & 0 & 2^k & 0 & \cdots & 0 & 0 \\ 
0 & 0 & 0 & 2^k & \cdots & 0 & 0 \\ 
\vdots   & \vdots   & \vdots   & \vdots   & \ddots & \vdots            & \vdots            \\
0 & 0 & 0 & 0 & \cdots & 2^k & 0 \\ 
0 & 0 & 0 & 0 & \cdots & 0 & 2^k \\ 
\end{pmatrix}
\begin{pmatrix}
A[0] \\ 
A[1] \\ 
A[2] \\ 
A[3] \\
\vdots \\
A[2^k - 2] \\
A[2^k - 1]
\end{pmatrix}
\\
&=
2^k
\begin{pmatrix}
A[0] \\ 
A[1] \\ 
A[2] \\ 
A[3] \\
\vdots \\
A[2^k - 2] \\
A[2^k - 1]
\end{pmatrix}.
\end{align*}

\eject \pdfpagewidth=8.5in \pdfpageheight=11in

It follows that the \textit{inverse FFT} is simply the same function, but run with $\omega^{-1}$ instead of $\omega$, and dividing out $m=2^k$.

\subsubsection{A Brief Note on Number Theory}

There is a slight variation on FTT that avoids using complex numbers completely. Complex numbers are often bad at precision and run slowly. There is a way to make FFT work over only the integers, which is pretty cool. Here's a brief description how.

We work $\pmod{p}$ for a suitable prime $p$. The reason we need $p$ to be prime is so that division works as we want. As explained in the above section on number theory, every nonzero number modulo $p$ has a unique inverse, so division works as expected. There are also integers $g$ such that $g^0, g^1, \dots, g^{p-2}$ run through all nonzero numbers $\pmod{p}.$ These are called \emph{primitive roots}. This means that modulo $p$ behaves a whole lot like the roots of unity that we want. More specifically, we want an integer $\omega$ such that the smallest integer $m$ that satisfies $\omega^m = 1$ is $m = 2^k,$ for some power of $2.$ So when can we find such an $\omega?$ By the discussion of primitive roots, this can happen precisely when $2^k | p-1.$ Then all we do is find a primitive root $g$, and take $\omega = g^{\frac{p-1}{2^k}}.$ Finding primitive roots is fast because math proves that there is always a small primitive root. So everything works just as you expect! The same exact code works, just put a $\pmod{p}$ at the end of every line and you're set.
